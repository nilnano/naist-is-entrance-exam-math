\documentclass[12pt, dvipdfmx, a4paper]{jsarticle}

\usepackage{amsmath, amssymb}
\usepackage{hyperref}
\usepackage{pxjahyper}

\usepackage[
  driver  = dvipdfm,
  truedimen,
  top     = 30truemm,
  bottom  = 25truemm,
  left    = 20truemm,
  right   = 20truemm
]{geometry}

\renewcommand{\headfont}{\bfseries}

\title{
  奈良先端科学技術大学院大学 情報科学研究科 \\
  平成 29 年度 博士前期課程入学者選抜試験 数学
}
\date{更新日: \today}

\begin{document}
\maketitle

\section*{解析}
\subsection*{第1回 7/7}
\subsubsection*{(1)}
級数 $\displaystyle \sum_{n = 1}^\infty \frac{n^2 + 2}{n^4 + 3n + 1}$ の収束・発散を判定せよ.

ただし,級数 $\displaystyle \sum_{n = 1}^\infty \frac{1}{n^p}$ は
$p > 1$ で収束し,$p \leq 1$ で発散することを用いてよい.

\subsection*{第1回 7/8}
\subsubsection*{(1)}
\[
  \int_0^\pi \frac{x \sin x}{1 + \cos^2 x} \: \mathrm{d}x
\]

$t = x - \pi$ として解け.


\clearpage


\section*{代数}
\subsection*{第1回 7/7}
\subsubsection*{(2)}
\[
  A = \begin{pmatrix} \alpha & 1 \\ 0 & \alpha \end{pmatrix}
\]

\begin{enumerate}
  \item 固有ベクトルを求めよ.
  \item $A^n$ を求めよ.
\end{enumerate}

\subsection*{第1回 7/8}
\subsubsection*{(1)}
\[
  A = \begin{pmatrix} a & b \\ c & d \end{pmatrix}
\]

$A^2 + 3A + 7E = O$ のとき,$ad - bc$ と $b + c$ を求めよ.

\end{document}
